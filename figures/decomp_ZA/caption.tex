\label{fig:ZA} Illustration that the unimodal assumption (UA) in \ashr can produce very different results from existing methods.
The figure shows, for a single simulated dataset, the way different methods decompose $p$ values (left) and $z$ scores (right) into a null component (dark blue) and an alternative component (cyan). In the $z$ score space the alternative distribution is placed on the bottom to highlight the differences in its shape among methods.
The three existing methods (\qvalue, \locfdr, \mixfdr) all effectively make the Zero Assumption, which results in a ``hole" in the alternative $z$ score distribution around 0.
In contrast the method introduced here (\ashr) makes the Unimodal Assumption -- that the effect sizes, and thus the $z$ scores, have a unimodal distribution about 0 -- which yields a very different decomposition. (In this case the \ashr decomposition is closer to the truth: the data were simulated under a model where all of the effects are non-zero, so the ``true" decomposition would make everything cyan.)