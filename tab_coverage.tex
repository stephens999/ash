\begin{table}
[!ht]
\begin{subtable}{\textwidth}
	\centering% latex table generated in R 3.1.2 by xtable 1.7-4 package
% Mon Apr 27 09:17:36 2015
\begin{tabular}{rrrrrrr}
  \toprule  & spiky & near-normal & flat-top & skew & big-normal & bimodal \\ 
  \midrule ash.n & 0.90 & 0.94 & 0.95 & 0.94 & 0.96 & 0.96 \\ 
  ash.u & 0.87 & 0.93 & 0.94 & 0.93 & 0.96 & 0.96 \\ 
  ash.hu & 0.88 & 0.93 & 0.94 & 0.94 & 0.96 & 0.96 \\ 
   \bottomrule \end{tabular}


	\caption{All observations. Coverage rates are generally satisfactory, except for the extreme ``spiky" scenario. This is due to the penalty term (\ref{eqn:penalty}) which tends to cause over-shrinking towards zero. Removing this penalty term produces coverage rates closer to the nominal levels for uniform and normal methods (Table \ref{tab:nopen}). Removing the penalty in the half-uniform case is not recommended (see Appendix).}
\end{subtable}

\begin{subtable}{\textwidth}
\centering% latex table generated in R 3.1.2 by xtable 1.7-4 package
% Mon Apr 27 09:17:35 2015
\begin{tabular}{rrrrrrr}
  \toprule  & spiky & near-normal & flat-top & skew & big-normal & bimodal \\ 
  \midrule ash.n & 0.93 & 0.94 & 1.00 & 0.94 & 0.95 & 0.98 \\ 
  ash.u & 0.86 & 0.88 & 0.93 & 0.91 & 0.94 & 0.94 \\ 
  ash.hu & 0.87 & 0.87 & 0.92 & 0.93 & 0.94 & 0.94 \\ 
   \bottomrule \end{tabular}


\caption{``Significant" negative discoveries. Coverage rates are generally satisfactory, except for the uniform-based methods in the spiky and near-normal scenarios,
and the normal-based method in the flat-top scenario. These results likely reflect inaccurate estimates of the tails of $g$ due to a disconnect between the tail of $g$ and the component distributions in these cases. For example, the uniform methods sometimes substantially underestimate the length of the tail of $g$ in these long-tailed scenarios, 
causing over-shrinkage of the tail toward 0.}
\end{subtable}

\begin{subtable}{\textwidth}
\centering% latex table generated in R 3.1.2 by xtable 1.7-4 package
% Mon Apr 27 09:17:35 2015
\begin{tabular}{rrrrrrr}
  \toprule  & spiky & near-normal & flat-top & skew & big-normal & bimodal \\ 
  \midrule ash.n & 0.94 & 0.94 & 0.94 & 0.86 & 0.95 & 0.96 \\ 
  ash.u & 0.93 & 0.93 & 0.93 & 0.84 & 0.95 & 0.95 \\ 
  ash.hu & 0.92 & 0.92 & 0.93 & 0.92 & 0.95 & 0.95 \\ 
   \bottomrule \end{tabular}


\caption{``Significant" positive discoveries. Coverage rates are generally satisfactory, except for the symmetric methods under the asymmetric (``skew") scenario. }
\end{subtable}

\caption{Table of empirical coverage for nominal 95\% lower credible bounds} \label{tab:coverage}

\end{table}